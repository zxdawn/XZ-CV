%-------------------------------------------------------------------------------
%	SECTION TITLE
%-------------------------------------------------------------------------------
\cvsection{研究经历}


%-------------------------------------------------------------------------------
%	CONTENT
%-------------------------------------------------------------------------------
\begin{cventries}

%---------------------------------------------------------
  \cventry
    {硕博阶段} % Job title
    {南京信息工程大学} % Organization
    {2017 - 至今} % Date(s)
    {导师:银燕} % Advisor
    {
      \begin{cvitems} % Description(s) of tasks/responsibilities
        \item {主持江苏省研究生科研与实践创新计划项目“估算闪电产生氮氧化物量”(KYCX20\_0922)}
        \item {参与国家自然科学基金重大项目“对流输送和闪电对大气成分垂直分布的影响及其机理研究”(91644224)}
        \item {参加国家自然科学基金重大项目“华北地区大气冰核观测和参数化及其对云降水的影响”(41590873)}
        \item {参与国家自然科学基金面上项目“内河船舶排放对空气污染的影响”(42075176)}
        \item {开发基于高分辨率化学模式 (WRF-Chem)的闪电NO$_x$反演算法}
      \end{cvitems}
    }

%---------------------------------------------------------
  \cventry
    {联培阶段} % Job title
    {荷兰皇家气象研究所 (KNMI)} % Organization
    {2021 - 2022} % Date(s)
    {导师: Ronald van der A} % Advisor
    {
      \begin{cvitems} % Description(s) of tasks/responsibilities
        \item {利用OMI/TROPOMI卫星观测,自主识别并追踪闪电产生的NO$_2$,计算其寿命及产率}
      \end{cvitems}
    }

% %---------------------------------------------------------
%   \cventry
%     {本科阶段} % Job title
%     {南京信息工程大学} % Organization
%     {2016 - 2017} % Date(s)
%     {导师:银燕} % Advisor
%     {
%       \begin{cvitems} % Description(s) of tasks/responsibilities
%         \item {使用卫星观测(MLS,FY-2)和再分析数据评估深对流对O$_3$垂直分布的影响}
%       \end{cvitems}
%     }

% %---------------------------------------------------------
%   \cventry
%     {本科阶段} % Job title
%     {南京信息工程大学} % Organization
%     {2013 - 2016} % Date(s)
%     {导师:王壮} % Advisor
%     {
%       \begin{cvitems} % Description(s) of tasks/responsibilities
%         \item {研究纳米氮化钛与溶解有机物共存时的生态毒理学效应}
%       \end{cvitems}
%     }

%---------------------------------------------------------
\end{cventries}
